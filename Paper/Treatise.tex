\documentclass[a4paper,10pt]{report}

% Language
\usepackage[UKenglish]{babel}    % define culture for rendering (eg., hyphenation rules)
\usepackage[T1]{fontenc}       % use utf8 fonts
\usepackage[utf8]{inputenc}      % input encoding is utf8
\usepackage{lmodern}             % use font Latin Modern

% Supreme boilerplating
\usepackage{etex}                % LaTeX 2.0 features on relics
\usepackage{microtype}           % Extremely anal typographic mode
\usepackage{url}                 % Ability to create URLs
\usepackage[hidelinks]{hyperref} % Makes intradocument links, eg., contents page
\usepackage{color}               % Adds colour features. Really.
\usepackage{graphicx}            % Adds image features
\usepackage{listings}
\usepackage{multirow}
\usepackage{longtable}
% Layout
\usepackage[a4paper]{geometry}   % In case I want to set margins etc.
\usepackage{setspace}            % For setting line spacing
\usepackage{parskip}
\usepackage{enumerate}           % Allows different types of numbering
\onehalfspacing
% Fix big margin-top on 'Chapter x'
\usepackage{titlesec}
\titlespacing*{\chapter}{0pt}{-50pt}{20pt}
\titleformat{\chapter}[display]{\normalfont\huge\bfseries}{\chaptertitlename\ \thechapter}{20pt}{\Huge}

% Actual Packages
\usepackage{fixme}

\usepackage{amsmath}             % Imports for maths things
\usepackage{amssymb}
\usepackage{amsthm}
\usepackage{empheq}              % For highlighting mathematics
\usepackage{tikz}
\usetikzlibrary{arrows,shapes,decorations.pathreplacing,decorations.pathmorphing,calc,patterns,scopes,trees,positioning,fit}
\usepackage{pgfplots}
\pgfplotsset{compat=1.7}
\pgfkeys{/pgf/number format/relative round mode=fixed}

% Bug Fixes
\usepackage{fixltx2e}            % LaTeX bugfixes (yes really)

\usepackage[authoryear, round, sort]{natbib} % References
% "fix the natbib spacing character" - Kevin Naudé
\makeatletter
\ifdefined \NAT@spacechar
	\def\NAT@spacechar{~}
\fi
\makeatother

\hypersetup{pageanchor=false}    % Wait until actual start of writing before making page anchors

\definecolor{nmmu_blue}{cmyk}{0.99,0.48,0,0.56}

%%%%%%%%%%%%%%%%%%%%%%%%%%%%%%%%%%%%%%%%%%%%%%%%%%%%%555
% Macros
\makeatletter
\renewcommand{\maketitle}{
	\newpage
	\null
	\begin{center}%
	\includegraphics[width=\textwidth]{nmmu.jpg}
	\vskip 2em%
		{\color{nmmu_blue}
			\rule{\textwidth}{2pt}
			\vskip 2em%
			
			{\huge \textbf{\@title} \par}%
		}
				\vskip 1em%
				{\Large Author: \it{\@author} \par}
				{\Large Supervisor: \it{Kevin Naudé}}
				\vskip 2em%
			\textcolor{nmmu_blue}{\rule{\textwidth}{2pt}}
	\end{center}%
	\par}
\makeatother

% Restarts numbering and sets format to 1, 2, 3...
\newcommand \startrealnumbers { 
	\pagenumbering{arabic}
	\setcounter{page}{1} 
	\hypersetup{pageanchor=true}
}

% Renders a bibliography on its own page, and adds a contents reference.
\newcommand \bibliographysection {
	\cleardoublepage
	\phantomsection
	\addcontentsline{toc}{chapter}{Bibliography}
	\bibliographystyle{authordate3}
	\bibliography{bibliography}
}

\newcommand \todo \fxwarning

\renewcommand \todo [1] { {\color{red}(TODO: #1)} } %Remove for final version
	

\newcommand \nakedchapter [1] {
	\chapter*{#1}
	\addcontentsline{toc}{chapter}{#1}
}


\newenvironment{myfigure}
	{ 	\begin{figure}
		\hrule \vskip 1em
		\begin{minipage}{\textwidth} \centering
	}
	{	\end{minipage}
		\vskip 1em \hrule
		\end{figure}
	}

\newenvironment{doublefig}
	{
		\begin{figure}
		\hrule \vskip 1em
	}
	{
		\vskip 1em \hrule
		\end{figure}
	}
\newenvironment{halffig}
	{ \begin{minipage}[b]{0.45\textwidth}\centering }
	{ \end{minipage}\hfill }

\lstset{xleftmargin=1em, basicstyle=\ttfamily}

\begin{document}

\title{A JIT-less, Register-Mapped, Statically-Typed Virtual Machine for Interpreters}
\author{Matthew Sainsbury}
\date{\today}

% Cover Page
\begin{titlepage}

		\maketitle
	
\end{titlepage}

\pagenumbering{roman}

% Prefaces
\nakedchapter{Acknowledgements}
	To all my loyal fans

\nakedchapter{Abstract}
	\todo{align this with project proposal}
	Conventional JIT-less high-level register virtual machines simulate virtual registers in random access memory (RAM) and load virtual registers from RAM whenever they are accessed \citep{caseregistervm}. This treatise investigates an alternative approach where virtual registers are mapped to physical registers, and instructions are dispatched not only on the opcode, but also on the operands of the instruction. To emulate a virtual instruction, the interpreter jumps to the appropriate code segment in a table of implementation code based on the instruction word to be executed. This table grows in a polynomial fashion with increased virtual registers, and therefore the performance balance between table size and register pressure is investigated.

\nakedchapter{Declaration of Own Work}


\tableofcontents

\nakedchapter{Glossary}
\begin{description}
	\item[Bytecode] Program code executed by a virtual machine.
	\item[Virtual Machine] A class of programs which translate code written for one architecture to another.
\end{description}


% Main document body
\chapter{Introduction}
	\startrealnumbers
	Interpreters are a popular way to implement programming languages because they allow programs to be portable, and, compared to natively compiled programs, more readily support advanced features like debuggers. 
	
	Virtual machines are commonly used in the implementation of interpreters. Virtual machines provide an abstraction layer between the interpreted program and the host machine. Being able to abstract away the details of each specific host machine affords interpreted languages their portability, as well as easing implementation by providing a ``stepping stone'' between the high-level ideas of a programming language and the low-level details of the processor.
	
	However, virtual machines add significant overhead to the execution of interpreted programs. Research effort has been expended to optimise virtual machines for interpreters, but much of this research was done on old processors which no longer reflect the structure of modern processors. For instance, modern processors have far larger cache sizes than the processors of previous generations.
	
	This treatise investigates a new implementation technique for virtual machine interpreters running on modern processors. Two ideas are presented: register mapping between virtual machine registers and real registers, and a new dispatch method which operates on full instruction words. The goal of this research is that these two ideas will allow a modern processor to handle the behaviour of a virtual machine.
	
	This section aims to establish the landscape of modern processor design and the problems which arise in virtual machine implementations, as well as describe the scope of this project.
	
	\section{Background}
		A virtual machine is a program that executes programs by emulating the instruction-level behaviour of a chosen machine. In some cases, virtual machines allow programmers to execute programs written for one kind of computer on a machine that has an unrelated architecture. In other cases, virtual machines run programs designed for theoretical machines that do not have a physical device to run them.
		
		This second kind of virtual machine may sound worthless to someone who has only been exposed to the more immediately visible applications of virtual machines, like playing games designed for obsolete game consoles\todo{just trying to make it more relatable}, but virtual machines like these serve as convenient and powerful design patterns in high-level programming language interpreters. These virtual machines are known as high-level virtual machines.
		
		A high level virtual machine runs programs for a machine which is imagined by its designer. Because it is not bound by the practical limitations of machines with real implementations, it may have more sophisticated features and other desirable properties which would be difficult to build into a real machine. High-level virtual machines can be considered to be an abstraction layer that hides the intricacies of any particular real machine. Virtual machines can be implemented on any physical architecture, resulting in the extremely useful property that code written for a virtual machine is very portable. The portable ``machine code'' that high-level virtual machines execute is usually called its \emph{bytecode}. 
		
		The primary use for high-level virtual machines is in high-level interpreted languages, such as C$^\sharp$, Java and Python. High-level language interpreters are typically implemented with a compiler that compiles the input source code into an intermediary bytecode, which is then interpreted on a high level virtual machine. This kind of interpreter executes programs much faster than interpreting the source code line-by-line.
		
		Virtual machine also offer software engineering benefits to implementing interpreters. Writing a compiler flexible enough to target many machines with vastly different designs is difficult, and especially difficult for modern, complex high level languages. A high-level virtual machine allows a compiler that targets a single, albeit virtual platform, while still retaining the portability of the system. It is easier to re-target a virtual machine for a new architecture than an entire compiler. Ease of compiler implementation is also increased by the high-level nature of the virtual machine, which can support instructions for complex tasks that would be difficult to compile down to machine code. From an architectural design point of view, virtual machines help keep interpreters organised internally, seperating the language and context analysis functions (in the compiler) from the host-specific implementation details (in the virtual machine). \citep{structureinterpreters}.
		
		Interpreters implemented with virtual machines are among the fastest kinds of interpreters\citep{modernarchvm}. However, they remain significantly slower than executing native code \citep{optimizingindirectbranch}. An important and widely used optimisation technique for virtual machine interpreters is Just-in-time (JIT) compilation. JIT is not explored in this research, but is a relevant and complementary technique. JIT tries to bring ``the best of both worlds'' together. Modern JIT interpreters execute an instrumented interpreter which gathers profiling data to determine which parts of the bytecode would most benefit from native execution. Once it has identified these sections of bytecode, it compiles them into the host machine's instruction set and executes them natively instead of interpreting them \citep{historyjit}. Most popular virtual machines utilise JIT compilation, such as Java's JVM, and C$^\sharp$'s Common Language Runtime (CLR).
		
		Although JIT compilation is a good strategy to improve the performance of virtual machines, it has disadvantages in certain use cases. An example is found in multi-instance programs, where a program is started many times, concurrently, so that multiple instances run as separate processes at the same time. A real-world case of such a program is a \todo{PHP thing} web server, which spawns several identical program instances to serve web clients asynchronously.
		
		Operating systems allow distinct processes to share memory under certain conditions. If two processes load the same resource into read-only memory blocks, the resource can be loaded into one memory location which is shared by both processes. This mechanism is automatic and transparent to the process, so long as the memory blocks concerned are read-only, and cannot be altered by the other process. An example of this in action is shown in \reffig{nativeprogram}, where two instances of the same program share a read-only code space in memory \citep{sharedcodepatent}. 
		
		When two instances of a JIT interpreter are started, they both load the interpreted program into memory. However, since JIT is a rewriting process, the two processes cannot share JIT data, because the operating system cannot negotiate transparent sharing of writeable memory blocks. Each interpreted process utilising JIT must do its own tracing and JIT compilation of the bytecode. In multi-instance programs, this results in the same code being compiled over and over again (\reffig{interpretedprogram}). JIT compilation is a hefty procedure, so multiple compilations of the same identical code is undesirable. 
		
		\begin{doublefig}
			\begin{halffig}
				\begin{tikzpicture}
				\draw (-0.2, 0.2) [dashed] rectangle (2.95, -3.2) node[below] at (1.375, -3.2) {Process 1};
				\draw (3.2, 0.2) [dashed] rectangle (6.35, -3.2) node[below] at (4.75, -3.2) {Process 2};
				\draw (0,0) [fill=white] rectangle (6.15,-1) node[midway] {Shared Interpreter Code Space};
				\draw (0,-1) rectangle (2.8, -3) node[midway] {Data Space}; 
				\draw (3.35,-1) rectangle (6.15, -3) node[midway] {Data Space}; 
				\end{tikzpicture}
				\caption{Two native programs}
				\label{fig:nativeprogram}
			\end{halffig}
			\begin{halffig}
				\begin{tikzpicture}
				\draw (-0.2, 0.2) [dashed] rectangle (2.95, -4.25) node[below] at (1.375, -4.25) {Process 1};
				\draw (3.2, 0.2) [dashed] rectangle (6.35, -4.25) node[below] at (4.75, -4.25) {Process 2};
				\draw (0,0) [fill=white] rectangle (6.15,-1) node[midway] {Shared Interpreter Code Space};
				\draw (0,-1) rectangle (2.8, -4.05) node[below] at (1.4,-1) {Data Space}; 
				\draw (3.35, -1) rectangle (6.15, -4.05) node[below] at (4.7,-1) {Data Space};
				\draw (0.15, -1.5) rectangle (2.65, -2.2) node[midway] {Bytecode Space};
				\draw (3.5, -1.5) rectangle (6, -2.2) node[midway] {Bytecode Space};
				\draw (0.15, -2.35) rectangle (2.65, -3.05) node[midway] {JIT Data};
				\draw (3.5, -2.35) rectangle (6, -3.05) node[midway] {JIT Data};
				\draw (0.15, -3.2) rectangle (2.65, -3.9) node[midway] {VM Data Space};
				\draw (3.5, -3.2) rectangle (6, -3.9) node[midway] {VM Data Space};
				\end{tikzpicture}
				\caption{Two interpreted programs}
				\label{fig:interpretedprogram}
			\end{halffig}
		\end{doublefig}
		
		Unfortunately, multi-instance applications have become much more common as CPUs increase in core count, and stagnate in clock frequency. Since the popular approach of JIT compilation is unsatisfactory in this use case, it is worthwhile to explore efficient virtual machine designs which do not utilise JIT compilation. This research falls into this area of interest. To facilitate dialogue in the space of non-JIT virtual machines, we present a background in the implementation of conventional virtual machines on modern architectures.
		
		\subsection{Conventional Virtual Machines}
			There are two common types of high-level virtual machine architectures: stack machines and register machines. Stack machines store temporary values in a stack data structure, while register machines store temporary values in storage ``boxes'' called registers. A stack machine's stack is used not just for temporary data, but serves a double purpose as the call stack. This is contrasted with a register machine, which maintains a dedicated call stack alongside the registers.
			
			Real register machines implement registers in high-speed memory hardware as part of their design. However, virtual machines have difficulty emulating high-speed registers because virtual machine registers are typically stored in the host's main memory.
			
			Most of a stack machine's instruction set consists of operations on the stack.  \reffig{stackprogram} shows an example of code written for a stack machine. Notice that the operands of the instructions (such as \texttt{add} and \texttt{divide}) are implicit, and so instructions for stack machines tend to be shorter. The complexity of individual instructions is major difference between stack machines and register machines. 
		
			\begin{doublefig}
				\begin{halffig}
					\begin{lstlisting}
push 14
push 8
add
push 7
divide
print
					\end{lstlisting}
					\caption{Stack machine program}
					\label{fig:stackprogram}
				\end{halffig}
				\begin{halffig}
					\begin{lstlisting}
mov regA, 14
mov reg1, 8
add regA, regB
mov regB, 7
divide regA, regB
print regA
					\end{lstlisting}
					\caption{Register machine program}
					\label{fig:registerprogram}
				\end{halffig}
			\end{doublefig}
			
			Unlike stack machine instructions, a register machine's instructions additionally specify the registers involved in the instruction. \reffig{registerprogram} shows the same program for a register machine. In the \texttt{add} instruction, the value of \texttt{regA} is added to \texttt{regB}, and the result is stored in \texttt{regA}.
		
			Each instruction in a program is encoded as a series of bytes called an \emph{instruction word}. The component of the instruction word which specifies the operation to be performed (eg., \texttt{add}) is called the \emph{opcode}. 
			
			In a stack machine, the instruction encoding is simple. Each opcode is assigned a unique instruction word. For instance, the program listed in \reffig{stackprogram} might be represented:
			
			\texttt{
				\begin{tabular} {| l | l |}
					\hline
					Encoding & Instruction \\
					\hline
					0x01 & push \\
					0x01 & push \\
					0x00 & add \\
					0x01 & push \\
					0x02 & divide \\
					0x03 & print \\
					\hline
				\end{tabular}
			}
			
			A register machine's instruction encoding is more complicated because the operands of the instruction need to be part of the instruction encoding. The conventional implementation uses bitfields. The first few bits are reserved for encoding the opcode, which follows the same format as with a stack machine. The rest of the bits are divided into segments, into which each operand is encoded. For example, consider a twelve-bit instruction word. The first four bits can be reserved for the opcode encoding, resulting in a maximum of 16 unique opcodes. The next four bits of the instruction word can encode the first operand as an enumeration, and the same for the last four bits which can encode the second operand. If an instruction only has one operand, the last four bits are undefined. This is illustrated in \reffig{bitfields}	
			
			\begin{myfigure}
				\texttt{
					\begin{tabular}{ | l | r | r |r | }
						\hline
						Meaning & opcode & dest & src \\ 
						\hline
						add regA, regB  & 0001 & 0000 & 0001 \\
						\hline
						add regD, regC & 0001 & 0011 & 0010 \\
						\hline
						divide regB, regA & 0010 & 0001 & 0000 \\
						\hline
						print regA & 0011 & 0000 & 0000 \\
						\hline
					\end{tabular}
				}
				\caption{Register instructions using bitfields}
				\label{fig:bitfields}
			\end{myfigure}
			
		\subsection{The Nature of Modern Processors}
			The landscape of processor design has changed significantly in the last thirty years. Rather than producing performance gains by arbitrarily increasing processor clock speeds, modern processor designers rely on trying to increase the per-cycle efficiency of their processors, using techniques such as instruction pipelining, larger and more complex cache systems, and speculative optimisation. These design issues caught the interest of academics and engineers in the 1980's \citep{modernprocessordesign}. Some of these design changes significantly alter the performance characteristics of virtual machine interpreters, and are briefly discussed in this section.
			
			\subsubsection{Instruction Pipelining}
			Early processors executed machine instructions one at a time. Every clock cycle, an instruction would be fetched, decoded, and executed. The result of this method is that activity on a processor would move through the architecture component-wise. Only one part of the process would be ``working'' at any time---for instance, when an instruction was being decoded, the entire execution component of the processor was unused.
			
			On a processor with pipelining, these functions can operate independently in parallel. An instruction can be fetched while another instruction is being decoded, and yet another instruction's operands are being fetched. A processor that can operate this way is said to have an \emph{instruction pipeline.} This results in far better utilisation of processor hardware, but introduces complexity in the processor. For instance, if one instruction writes to a memory address, and later instruction reads from the same address, the second instruction cannot complete the operand fetch stage until the first instruction has been executed. A situation where an instruction has to wait for another to complete is known as a \emph{pipeline stall,} and is a subject of much academic writing and engineering effort.
			
			\subsubsection{Cache Arrangement}
			Modern processors contain portions of high-speed memory which act as a cache to the main memory of the machine. The behaviour of this cache is an important factor for processor performance, because main memory is quite slow compared to the speed of a CPU. Modern caches are orders of magnitude larger than processors twenty years ago. 
			
			Most modern processors contain a series of specialised caches. For instance, a fourth-generation Intel Core processor has three grades or \emph{levels} of cache, and make a distinction between cache for instructions and cache for program data. \reffig{cachenumbers} shows the arrangement of cache on such a processor. As can be seen, L1 cache is much faster, but much smaller than L2 cache. The latency in L1 instruction cache is not shown because instruction fetch is transparent to programs, but it does influence branch prediction (see later). It is advantageous for a program if most of its application data can fit in L1 data cache, and most of its code can fit in L1 instruction cache. However, traditional virtual machines have not been designed to take advantage of the large amount of cache available on modern processors.
			
			\begin{myfigure}
				\begin{tabular}{ | l | l | l | l | }
					\hline
					Level & Type & Size & Latency \\ 
					\hline
					\multirow{2}{*}{L1} & Instruction & 32KB & N/A \\
					& Data & 32KB & 4 cycles \\
					\hline
					L2 & Data & 256KB & 11 cycles \\
					\hline
					L3 & Data & Varies & Varies \\
					\hline
				\end{tabular}
				\caption{Cache on 4th-Gen Intel Core CPUs \citep{optimisationreference}}
				\label{fig:cachenumbers}
			\end{myfigure}
			
			\subsubsection{Branch Prediction \& Branch Target Prediction}
			Branch prediction in a process refers to the processor's ability to continue the pipelining process during a branch instruction. Ordinarily, the location of the next instruction is known to the processor---it is simply the instruction adjacent to the previous one. If the previous instruction was an instruction to branch (go somewhere else in the program), the picture becomes more complex. There are three types of branches worth mentioning:
			
			\begin{description}
				\item[Unconditional Branch:] This instruction always causes the processor to jump to the same location. The target of the branch is known as soon as the instruction is decoded.
				\item[Conditional Branch:] This branch type branches to a constant location if a condition is true. For instance, if a register is zero (``branch if zero''). The location of the next instruction is only known once the branch instruction is executed.
				\item[Indirect Branch:] A branch whose branch location is based on program data, whether in memory or in a register, or both. 
			\end{description}
			
			An unconditional branch is the easiest branch to pipeline, since the location of the next instruction is encoded in the branch instruction itself. The next instruction can be fetched as soon as the branch instruction is decoded. 
			
			A conditional branch is harder to pipeline because it is not known whether the condition is true or false before the instruction is executed, and thus what the next instruction should be. To prevent pipeline stalls on conditional branch instructions, modern processors implement a \emph{branch predictor}. The branch predictor will use some heuristic to predict the conditional, and load the most likely next instruction. A pipeline stall will only occur if the prediction was wrong---this delay is known as the \emph{misprediction penalty}.
			
			An indirect branch is the worst type of branch for a pipelined processor. It is very hard to predict because the branch location could depend on any data anywhere in main memory or in registers. This influence that data has on the behaviour of the program is known as \emph{data dependance} and is to be avoided whenever possible. Modern processors implement a \emph{branch target predictor} in an attempt to reduce the number of pipeline stalls. A branch target predictor is not to be confused with a branch predictor---a branch predictor predicts whether the branch will be taken, whereas a branch target predictor tries to predict where the branch will lead. A conditional branch can only lead two places---the next instruction, or the constant branch location.
			
			One common branch target predictor is a \emph{branch target buffer} (BTB). A branch target buffer maintains a table storing the last target location of each indirect branch encountered. Thus a BTB makes the assumption that an indirect branch target is unlikely to change  \citep{yeti}. This strategy works well if an indirect branch always branches to the same location, but is no help if it is always different.
			
			Indirect branches are a big issue in virtual machine design, since an indirect branch is unavoidable during the dispatch process of a virtual machine \citep{structureinterpreters}. An indirect branch may be unavoidable, but it may be possible to significantly increase the dispatch performance of a virtual machine by increasing the likelihood that indirect branches are predicted correctly by a branch target predictor.
			
			These concepts form the background to this research, and the problems which motivate it.
	
	\section{Problem Description}
		Most physical machines are register machines. It is thought that the use of real machine registers to store the values of the virtual machine registers will help to make the operation of virtual machine programs more transparent to the physical CPU by reducing data-dependant fetches from main memory, where virtual machine registers are usually stored. Unfortunately, unlike virtual registers that are implemented in memory, real registers cannot be accessed by ``reference'' \citep{caseregistervm}. This difference complicates the implementation of a register-mapped virtual machine.
		
		Despite these difficulties, it is worthwhile to investigate designs of JIT-less register virtual machines that attempt to narrow the divide between the host and guest architecture.
	
		\todo{Do I need more?}
		
	\section{Purpose and Scope}
		The goal of this project is to investigate the feasibility of JIT-less register virtual machines for interpreters running on a modern architecture. To this end, a high-level register virtual machine will be implemented with some unique architecture features that have not been investigated or thought feasible previously.
		
		A novel dispatch process will be investigated that seeks to allow the use of physical machine registers in the implementation of virtual machine registers. To circumvent the problem of referencing real machine registers, the virtual machine will dispatch not only based on the opcode of each instruction, but on the instruction as a whole. In other words, an implementation will exist for every combination of operands in each virtual machine instruction.
	
		An instruction set architecture will be designed which will be tailored for such a dispatch method. It is anticipated that having implementations for every combination of operands in each virtual machine instruction will cause the virtual machine implementation to become very large, and so the design of the instruction set architecture will focus on keeping the number of unique instructions low.
		
		The virtual machine will target the Intel 64 host machine architecture.
		
		\subsection{Limitations}
			As the virtual machine implementation becomes very large, it will become unsuitable for operation on machines with small L1 and L2 instruction caches, because large portions of the virtual machine implementation will not remain in the instruction cache. The result is a virtual machine that is actually slower than conventional virtual machines. We will consider only modern processors with reasonably large first-level cache, for instance, the current Intel Core i7 range of processors containing 32KB of L1 instruction cache.
			
			Assuming any virtual machine register can be used as an operand for any instruction, the size of the virtual machine implementation proportional to:
			
			\[
				\sum_{o~\in~opcodes} r^n : 
					\begin{array}{l}
						n \coloneqq \text{number of operands for $o$} \\
						r \coloneqq \text{number of virtual machine registers}
					\end{array}
			\] 
			
			Every additional virtual machine register increases the size of the virtual machine by a polynomial factor. At some point, any performance gain received by exposing more physical registers will be negated by increasing incidence of cache misses. This is a source of a practical limitation on the number of registers in the virtual machine's architecture.
			
				
	\section{Overview of Treatise Structure}
		\todo{Will have to wait until the rest of the treatise exists for this! Tip: Design this for someone who doesn't feel like reading the whole thing. Make it easy to find specific, interesting sections.}

% 10-12 pages
\chapter{Problem Exposition}
	\section{Literature Review}
		\cite{structureinterpreters} reflect on why spending research and engineering effort on optimising interpreters is a worthwhile endeavour. Creating a machine abstraction layer between original source code and host machine architecture greatly simplifies the writing of compilers that target many platforms. Without such an abstraction layer for each host machine, a separate version of the compiler targeting that platform must be developed. Interpreters allow compilers to be more versatile with little effort. In a high-level language interpreter utilising a virtual machine, only the interpreter portion needs to be rewritten for different platforms, (or perhaps not even the interpreter if a high-level language is used to implement the interpreter) and the compiler remains the same. Interpreters tend to be simple compared to compilers targeting machine code, so this advantage is quite significant. Compared to native execution, interpreters also offer opportunity for more powerful usability features like debuggers, and have faster compilers, improving the write-compile-test cycle of modern software development methods.
		
		A good design of a high-level virtual machine is one that is both easy to compile code for (sophisticated) and allows for fast interpretation (low overhead).
		
		The feature that is present in all virtual machine interpreters is a dispatch process. Dispatch refers to the process of fetching, decoding, and starting the execution of a virtual machine instruction. Dispatch technique can have the biggest effect on the performance of a virtual machine.
		
		The most efficient dispatch is direct threaded code. ``Threading'' in this case does not refer to ``multithreading'', but instead to a much older type of code structuring technique. In this method, a virtual machine instruction is actually the address of the code that implements that instruction. This is illustrated in \reffig{directthreading}. Notice the line `\texttt{jmp *pc++}'. This is an indirect branch. An indirect branch is an inescapable consequence of the dispatch process. Direct threading is efficient because, although it contains an indirect branch, there is little other overhead.
		
		\begin{myfigure}
				\begin{lstlisting}
start:
  pc = bytecode
  jmp *pc++

bytecode:
  &add
  &divide
  &print

add:
  //implementation
  jmp *pc++

divide:
  //implementation
  jmp *pc++

print:
  //implementation
  jmp *pc++

				\end{lstlisting}
				\caption{Direct Threading Dispatch}
				\label{fig:directthreading}
		\end{myfigure}
		
		The reason that the jump instruction occurs at the end of each implementation is that duplicating the indirect jump increases the utilisation of a branch target prediction, since there is an entry in the branch target buffer for each indirect branch instruction. If there was one common jump instruction, the BTB would always predict that the next instruction is the same as the previous one, which is not usually the case. This will be better explained further in this chapter.
		
		Other techniques for improving the accuracy of branch target prediction have been investigated, and some more interesting ones are summarized by \cite{optimizingindirectbranch}. 
		
		One interesting finding of that paper was that branch prediction in loops could be improved if no duplicate instructions occur in the loop. Branch prediction accuracy can be 100\% in that case, because the next instruction can always be predicted based only on the current instruction. \citeauthor{optimizingindirectbranch} suggested a counterintuitive technique of duplicating instructions to allow compilers to ensure that each instruction is only used once in a loop.
		
		Another interesting technique investigated by \citeauthor{optimizingindirectbranch} is the use of `superinstructions' which are groups of instructions with a single opcode. By combining instructions into more complex instructions which are used less often, this technique can reduce branch prediction failures.
		
		Traditionally, the high level virtual machines implemented in interpreters are stack machines. Stack machines are used over register machines for a few reasons. Instructions for stack machines are smaller, because instruction operands are implicit. This means that instruction fetching is faster. It is easier to compile code for a stack machine than a register machine because the compiler does not need to implement a register allocator \citep{caseregistervm}. However, \cite{stackregistershowdown} found that an efficient register virtual machine can execute benchmarks 32\% faster than an analogous stack machine. 
		
		\cite{caseregistervm} mentions that the first successful virtual machine interpreter ran stack-based P-code for Pascal. This was the first in a large family of stack-based bytecodes for popular languages like Smalltalk, Java and the Common Intermediate Language (CIL). This suggests that there might be historic factors involved in the prevalence of stack based virtual machines.
		
		\cite{fastjava} believes that interpreters are not well designed to suit modern architectures features like pipelining, branch prediction and caches. They also provide encouragement by saying, ``if interpreters can be made much faster, they will become suitable for a wide range of applications that currently need a JIT.''
		
		\subsection{Relevant Previous Work}
		The practice of mapping virtual machine resources to host machine registers has been investigated for stack machines by \cite{stackcaching}. Ertl measured the performance of stack machines which kept the top few elements of the stack in registers. He found that direct mapping of stack positions to machine registers was only useful for the case where just the top element of the stack was cached in a register. Further caching resulted in too many load-store instructions to shuffle register values around when positions on the stack change.
		
		Ertl also attempted a technique he calls ``dynamic stack caching'' where stack elements are cached in whatever register is convenient, and an internal state machine remembers the mapping between stack context and registers. The technique employs multiple versions of code to emulate each virtual machine instruction; the version which is executed depends on the current state of the state machine (register-stack state mapping). Ertl found that this technique halves the cost of operand fetching for each additional register involved in the caching, but adds an instruction dispatch cost because of the increased number of states. A possible explanation for this is the increase in dispatch complexity, and increased cache miss rate due to the large number of states.
		
		In the same paper, Ertl writes the first mention of the idea of achieving register mapping between a register virtual machine and a host register machine through a table of implementations for each combination of opcode and operands. However, the idea is quickly dismissed, as it would ``cause code explosion, and will probably suffer a severe performance hit on machines with small first-level caches.'' The R4000 MIPS processor was a contemporary example, having only 8 KB of level one (L1) instruction cache.
	
		The state of cache availability has changed for modern processors. Consequently, this approach deserves further investigation. Modern processors have significantly larger cache sizes. For example, 4th generation Intel Core processors have 32 KB of L1 instruction cache \citep{optimisationreference}, four times as much as the R4000. They also have much larger second and third level caches.
		
		\todo{Should I spend much more effort on this subsection?}
	
		
	\section{Difficulties}
		Interpreters spend most of their execution time on instruction dispatch \citep{modernarchvm}. The main contributor is usually an indirect branch to the implementation code for the currently executing virtual machine instruction \citep{optimizingindirectbranch}. Every interpreter has at least one indirect branch in dispatch code \citep{modernarchvm}. Because these indirect branches are in the instruction dispatch portion of the interpreter, they are executed for every virtual machine instruction. This closely couples interpreter speed to efficiency of branching.
		
		Modern architectures tend to have long pipelines and perform branch target prediction to load the target of an upcoming branch before the branch instruction is executed. Branch target misprediction is very expensive in these architectures, because the execution of a branch happens late in the pipeline but affects the start of the pipeline \citep{optimizingindirectbranch}.
				
		Conventional interpreter designs that were optimal on older architectures perform poorly on modern pipelined architectures because branch prediction accuracy is a big factor on performance on these architectures. These interpreter designs hide the logic governing branching patterns from the branch predictor. All the host processor sees is branches that depend on data, and is not aware that this data stream represents a program. The machine abstraction layer causes the virtual machine to inherit the weaknesses of deep pipelining from the modern host processor, while abstracting away opportunities of dynamic optimisation in the modern host.
		
		\cite{yeti} explains this by saying ``the control transfer from one body to the next is data dependent on the sequence of instructions making up the virtual program.'' This is an atypical scenario for modern processors, and they are not designed optimally for this task.
		
		Consider the virtual machine dispatch process on a modern processor with a branch target buffer (BTB). When a dispatch's indirect branch is encountered for the second time, BTB predicts the target will remain the same. However, this is not the case for the dispatch's branch, because the indirect branch is based on the current virtual machine opcode, which could be any of the instructions in the virtual machine instruction set. For each of these opcodes, the interpreter will branch to a particular section of code to emulate the virtual machine instruction. In other words, \emph{the BTB prediction will be accurate only if the opcode being dispatched is the same as the previous dispatch}. Because it is not very likely that the virtual machine will see many of the same opcode over and over, the BTB fails to predict the branch target most of the time. Opcodes ``fight'' for the single target record in the BTB that corresponds with the indirect branch in the dispatch code. It is now clear why the illustration of a conventional interpreter in \reffig{directthreading} duplicates the indirect branch per instruction.
		
		\begin{myfigure}
			\begin{tabular}{c c}
				{
				\begin{lstlisting}
start:
  pc = bytecode
  jmp *pc++

bytecode:
  &add
  &add
  &divide
  &add
  &print

add:
  //implementation
  jmp *pc++

divide:
  //implementation
  jmp *pc++

print:
  //implementation
  jmp *pc++
				\end{lstlisting}
			} & 
			{
				\begin{tikzpicture}
					\draw (0,7) -- (1, 7);
				\end{tikzpicture}
			}
			\end{tabular}
			\caption{Illustration of Indirect Branch Problems in Interpreters}
			\label{fig:interpreterbtb}
			\todo{Can't figure out drawing arrows on code}
		\end{myfigure}
		
		\cite{structureinterpreters} found that the branch misprediction penalty consumes up to half of the execution time on some interpreters. This high proportion is a result of the fact that most of an interpreter's time is spent in the instruction dispatch process.
		
		One way to get around this problem is by duplicating the indirect branch instruction, resulting in a separate BTB entry for each branch instruction \citep{fastjava}. This can be done by duplicating the dispatch code at the end of each virtual instruction implementation, instead of jumping back to a common dispatch code. In contrast, the traditional focus in branch optimisation was to make the code path as short as possible.
		
% 10-12 pages
\chapter{Solution Design}
	In this section the design of the virtual machine will be discussed. \todo{more}
	
	The overarching design goal for this virtual machine is to enable register mapping between virtual machine and host machine.
	
	In conventional virtual machines, an implementation for a virtual machine instruction will look like \reffig{operandfetch}. Notice how the virtual registers are implemented as an array in memory, and that the index of the virtual register is encoded into the instruction word (a bitfield, as explained earlier). If we are to store the virtual registers in real registers, we cannot access the register using an index.
	
	\todo{expand on dispatching, and virtual machine implementation in general}
	
	\begin{myfigure}
		\begin{lstlisting}
add(instr):
  ai = getSourceRegisterIndex(instr) //decode
  bi = getDestinationRegisterIndex(instr) //decode
  a = registers[ai]
  b = registers[bi]
  registers[bi] = a + b
  dispatch()
		\end{lstlisting}
		\caption{Operand Load/Store in Conventional Implementations}
		\label{fig:operandfetch}
	\end{myfigure}
	
	Instead, we will duplicate the implementation code for every combination of registers, as in \reffig{dupimplementation}. Notice how the implementation is much simpler (we have managed to remove all the operand fetch code from before), although there is a significantly larger amount of code. By hard-coding the registers instead of dynamically accessing them, we are able to circumvent the restriction that real registers cannot be dynamically indexed.
	
	\begin{myfigure}
		\begin{lstlisting}
add_0_0:
  regA = regA + regA
  dispatch()

add_0_1:
  regA = regA + regB
  dispatch()

add_0_2:
  regA = regA + regC
  dispatch()
  ...

add_1_0:
  regB = regB + regA
  dispatch()

add_1_1:
  regB = regB + regB
  dispatch()
...		
		\end{lstlisting}
		\caption{VM Implementation Using Code Duplication}
		\label{fig:dupimplementation}
	\end{myfigure}
	
	This method raises the problem that the virtual machine implementation code might become extremely large. If this happens, it is less likely that a particular part of the code will be found in the processor's cache, and will perform badly. We repeat the equation presented in the limitations discussion which characterizes the size of the implementation:
	
	\[
	\sum_{o~\in~opcodes} r^n : 
	\begin{array}{l}
	n \coloneqq \text{number of operands for o} \\
	r \coloneqq \text{number of virtual machine registers}
	\end{array}
	\] 
	
	There are several things we can do to attempt to reduce the inevitably large size of the implementation. The most obvious thing we can do is reduce the number of instructions supported by the VM architecture. This is more possible in high-level virtual machines than real machines, because the instructions tend to be more sophisticated. The second thing we can do is to reduce the number of registers exposed through the VM architecture. Because the size of the implementation has a polynomial relation to the number of exposed registers, this will probably have the largest effect. Our last attempt to reduce the size of the implementation will be to prohibit certain trivial combinations of operands for certain instructions. For instance, the instruction `\texttt{sub regA, regA}' is useless because the result will always be the same as `\texttt{mov regA, 0}'. Similar arguments lead us to remove combinations like `\texttt{xor regA, regA}', `\texttt{and regA, regA}' and so on. Removing these instructions from our specification means we can remove their implementations, because every combination of operands has its own implementation in our virtual machine.
	
	Since our virtual machine is designed to be run as a user process under an operating system, we should include instructions that help the program interact with its environment under the operating system. We call these instructions ``pragmatic'' instructions.
	
	With all these constraints and goals in mind, we can design the virtual machine architecture.
	
	\section{Virtual Machine Architecture Design}
		When designing the virtual machine architecture, it will be beneficial to remember this heuristic by \cite{structureinterpreters}: ``Well-designed VMs are tailored for both easy compilation from the source language and fast interpretation on real machines.'' \todo{too fluffy?}
		
		Our virtual machine will be developed for a static language. Consequently, we do not have to do type checking in the virtual machine. Registers in our virtual machine will notionally be one of two types---64-bit integer or pointer. The machine will have six general purpose registers, each of which can contain either a pointer or an integer. However, it is illegal to treat a pointer as an integer or vice versa. For this reason, a general-purpose register $g_i$ can be thought of as two distinct registers: an integer $r_i$ and a pointer $p_i$, only one of which is ``alive'' at any time. In fact, on real architectures that make a distinction between pointers an integers, these two ``virtual registers'' will be mapped to two distinct physical registers, so our strange convention is justified.
		
		Because this is a high-level virtual machine, pointers work like high-level languages in which pointer arithmetic is transparent. It is illegal to do arithmetic on pointers. This decision was made to make the interpreted language memory safe.
		
		There are two notional registers which are not visible to the programmer at all. These are \texttt{pc}, the virtual machine instruction pointer, and \texttt{fp} the stack frame pointer.
		
		Instructions are sixteen bits long. Unlike the majority of virtual machines, the instruction words are not bitfields. Since every unique instruction word corresponds to its own implementation code, and dispatch occurs over the entire word, bitfields serve no purpose, and actually make token threading more difficult because instructions with bitfields are not sequential.
		
		An instruction is sometimes followed by a sixteen-bit immediate value. This immediate value can be either a sixteen-bit constant, or a relative displacement to a position elsewhere in the program. This interpretation depends on the meaning of the preceeding instruction.
		
		A pointer can refer to one of three data structures: an object, an array of integers, or an array of pointers. An object is a collection of variables of different types in an order specified by an object prototype. The prototype is found in the program's bytecode and has the format specified in \reffig{objproto}. Since the virtual machine is statically typed, it is necessary to keep track of the type of each member in an object, and the prototype serves this purpose. However, this is not necessary for arrays, since they are homogenous---either all pointers or all integers.
		
		\begin{myfigure}
			\begin{tabular}{|l|l| }
				\hline
				Word & Meaning \\
				\hline
				0 & Number of fields ($n$) \\
				\hline
				$i \in \{1, 2, ...\}$ & Bitmap: bit $b$ is set if field $i\times64 + b$ is a pointer \\
				\hline
			\end{tabular}
			\caption{Object Prototype Definition}
			\label{fig:objproto}
		\end{myfigure}
		
		\subsection{Function Call and Return}
		The virtual machine's instruction set provides a high-level interface for calling functions through the \texttt{call} and \texttt{ret} instructions, which call and return from a function respectively. These instructions maintain a call stack and local variable space.
		
		When a function is called, a local variable space is created for the function. A function defines how much space it needs for local variables, and the types of those locals, in a \emph{function prototype}. 
		
		Function prototypes work similarly to object prototypes, as shown in \reffig{funcproto}. A function prototype begins with an eight-byte field which contains the number of locals the function needs. The corresponding amount of space is allocated for accessing these variables through \texttt{getl}, \texttt{setl}, \texttt{getlp} and \texttt{setlp}. The next few eight-byte words contain bitmaps which store the type of each local. Immediately following the appropriate number of these bitmap words, the bytecode for the function begins.
		
		\begin{myfigure}
			\begin{tabular}{|l|l| }
				\hline
				Word & Meaning \\
				\hline
				0 & Number of locals ($n$) \\
				\hline
				$i \in 1, 2, ...$ & Bitmap: bit $b$ is set if local $i\times64 + b$ is a pointer \\
				\hline
			\end{tabular}
			\caption{Function Prototype Definition}
			\label{fig:funcproto}
		\end{myfigure}
		
		When a function is called, a new stack frame is initialised for the function. Registers $r_1$ through $r_4$, $pc$ and $fp$ are saved automatically, while registers $r_0$ and $r_5$ are used for return values. A number of locals from the caller are copied into the callee's local space, as determined by an immediate value passed to the \texttt{call} instruction.
		
		A bytecode program is expected to start with a function prototype, which is automatically called by the virtual machine at execution start, much like C and its \texttt{main()} method. When this method calls \texttt{ret}, the program terminates.
		
	\section{Implementation Details}
		Because we need to map machine registers to virtual ones, our virtual machine must be able to manage registers directly. For this reason, the virtual machine is written mostly in Intel 64 assembly. It also makes use of standard C utility libraries for easy interface with the operating system.
		
		The basic structure of the assembly looks like the code sample in \reffig{vmstructure}. The vectors section contains an array of references to implementations for each instruction and operand combination. This array can be indexed using the instruction itself. In the code sample below, the instruction \texttt{0x0000} corresponds to \texttt{add r0, r0}, the instruction \texttt{0x0002} corresponds to \texttt{add r0, r2} and so on. Again, unlike traditional instruction sets, the actual value of the instruction has no meaning, and is just a unique index into this array.
		
		\begin{myfigure}
			\begin{lstlisting}
vectors:
dq add_0_0
dq add_0_1
dq add_0_2
...
dq add_1_0
...
dq sub_0_0
...

add_0_0:
add rbx, rbx
lodsw
jmp [vectors+rax*8]

add_0_1:
add rbx, rcx
lodsw
jmp [vectors+rax*8]
...
sub_0_0:
sub rbx, rbx
lodsw
jmp [vectors+rax*8]
...
			\end{lstlisting}
			\caption{Basic Structure of Virtual Machine}
			\label{fig:vmstructure}
		\end{myfigure}
		
		The complete virtual machine is much too large to write by hand---it is over 15 000 lines long. Instead, an automated script written in Ruby is provided with templates for each instruction. These templates are then expanded into the full source code for the virtual machine. Templates use a version of the Liquid templating language which has been modified for berevity. \reffig{addtemplate} and \reffig{shrtemplate} shows an example of one of these templates. Text surrounded by \texttt{\{ \}} is evaluated as a Ruby expression, and \texttt{\{\% \%\}} surrounds Liquid programming constructs like if statements.
		
		\begin{doublefig}
			\begin{halffig}
				\begin{lstlisting}
add {r[i]}, {r[j]}
				\end{lstlisting}
				\caption{Add Instruction Template}
				\label{fig:addtemplate}
			\end{halffig}
			\begin{halffig}
				\begin{lstlisting}

  xchg rcx, {r[j]}

  shr {r[i]}, cl

  xchg rcx, {r[j]}

				\end{lstlisting}
				\caption{Shift Right Instruction Template}
				\label{fig:shrtemplate}
			\end{halffig}
		\end{doublefig}
		
		The name \texttt{r} referred to in the templates is an array containing the mapping between virtual registers and real registers. For our implementation, the mapping is as follows:
		
		\begin{tabular}{|l|l|}
			\hline
			From & To \\
			\hline
			0 & 'rbx' \\
			\hline
			1 & 'rcx' \\
			\hline
			2 & 'rdx' \\
			\hline
			3 & 'r8' \\
			\hline
			4 & 'r9' \\
			\hline
			5 & 'r10' \\
			\hline
			'pc' & 'rsi' \\
			\hline
			'fp' & 'rbp' \\ 
			\hline
		\end{tabular}
		
		The Ruby script defines the \texttt{i}, \texttt{j} and \texttt{k} variables for each template expansion to represent the operands to the instruction, and also appends a dispatch template to the end of each implementation.
			
			
		\todo{Something here}
		
		In our implementation, every data structure begins with a 64-bit header field with an organisation described in \reffig{objheader}. 
		
		\begin{myfigure}
			\begin{tabular}{|l|c| c| c|c|c|}
				\hline
				Bit Position & 61--4 & 3 & 2 & 1 & 0 \\
				\hline
				Meaning & numeric data & GC1 & GC0 & OB & PA \\
				\hline
			\end{tabular}
			\newline
			\begin{description}
				\item[GC1:] Reserved for garbage collector
				\item[GC0:] Reserved for garbage collector
				\item[OB:] Bit set if structure is an object type
				\item[PA:] Bit set if structure is an array of pointers
			\end{description}
			
			\caption{Data Structure Header Definition}
			\label{fig:objheader}
		\end{myfigure}
		
		If the header describes an integer array, \todo{Continue}

\chapter{Evaluation Methods and Results}

% Last chapter
\chapter{Conclusion}
	
	\section{Opportunities for Future Development}
	
	\section{Reflection}

% Bibliography
\bibliographysection

\end{document}