\documentclass[a4paper,10pt]{report}

% Language
\usepackage[UKenglish]{babel}    % define culture for rendering (eg., hyphenation rules)
\usepackage[T1]{fontenc}       % use utf8 fonts
\usepackage[utf8]{inputenc}      % input encoding is utf8
\usepackage{lmodern}             % use font Latin Modern

% Supreme boilerplating
\usepackage{etex}                % LaTeX 2.0 features on relics
\usepackage{microtype}           % Extremely anal typographic mode
\usepackage{url}                 % Ability to create URLs
\usepackage[hidelinks]{hyperref} % Makes intradocument links, eg., contents page
\usepackage{color}               % Adds colour features. Really.
\usepackage{graphicx}            % Adds image features

% Layout
\usepackage[a4paper]{geometry}   % In case I want to set margins etc.
\usepackage{setspace}            % For setting line spacing
\usepackage{enumerate}           % Allows different types of numbering
\onehalfspacing

% Actual Packages
\usepackage{fixme}
\usepackage[authoryear, round, sort]{natbib} % References
\usepackage{amsmath}             % Imports for maths things
\usepackage{amssymb}
\usepackage{amsthm}
\usepackage{empheq}              % For highlighting mathematics
\usepackage{tikz}
\usetikzlibrary{arrows,shapes,decorations.pathreplacing,decorations.pathmorphing,calc,patterns,scopes,trees,positioning,fit}
\usepackage{pgfplots}
\pgfplotsset{compat=1.7}
\pgfkeys{/pgf/number format/relative round mode=fixed}

% Bug Fixes
\usepackage{fixltx2e}            % LaTeX bugfixes (yes really)
% "fix the natbib spacing character" - Kevin Naud�
\makeatletter
\ifdefined \NAT@spacechar
	\def\NAT@spacechar{~}
\fi
\makeatother

\hypersetup{pageanchor=false}    % Wait until actual start of writing before making page anchors


%%%%%%%%%%%%%%%%%%%%%%%%%%%%%%%%%%%%%%%%%%%%%%%%%%%%%555
% Macros

% Restarts numbering and sets format to 1, 2, 3...
\newcommand \startrealnumbers { 
	\pagenumbering{arabic}
	\setcounter{page}{1} 
	\hypersetup{pageanchor=true}
}

% Renders a bibliography on its own page, and adds a contents reference.
\def \bibliographysection {
	\cleardoublepage
	\phantomsection
	\addcontentsline{toc}{chapter}{Bibliography}
	\bibliographystyle{authordate3}
	\bibliography{bibliography}
}

\newcommand \todo \fxwarning

\begin{document}

\title{A JIT-less, Register-Mapped, Static-ISA Virtual Machine for Many-Instance Applications}
\author{Matthew Sainsbury}
\date{\today}

% Cover Page
\begin{titlepage}

		\maketitle
	
\end{titlepage}

\pagenumbering{roman}

% Prefaces
\nakedchapter{Acknowledgements}
	To all my loyal fans

\nakedchapter{Abstract}
	\todo{align this with project proposal}
	Traditional JIT-less high-level register virtual machines simulate virtual registers in random access memory (RAM) and load virtual registers from RAM whenever they are accessed \citep{caseregistervm}. This treatise investigates an alternative approach where virtual registers are mapped to physical registers, and instructions are dispatched not only on the opcode, but also on the operands of the instruction. To emulate a virtual instruction, the interpreter jumps to the appropriate code segment in a table of implementation code based on the instruction word to be executed. This table grows in a polynomial fashion with increased virtual registers, and therefore the performance balance between table size and register pressure is investigated.

\nakedchapter{Declaration of Own Work}


\tableofcontents

\nakedchapter{Glossary}
\begin{description}
	\item[Word] Definition
\end{description}


% Main document body
\chapter{Introduction}
	\startrealnumbers
	In this treatise, the investigation and analysis of a design for a new type of virtual machine for interpreters that supports efficient execution of many-instance programs will be investigated. Many-instance programs are programs which perform many process forking operations. This section aims to establish why such an endeavour is worthwhile by presenting a brief outline of the many existing designs for high level virtual machines
	
	\section{Background}
		High level language interpreters are typically implemented with a compiler that compiles the input source code into an intermediary bytecode, which is then interpreted on a high level virtual machine. High level virtual machines are not simply system virtual machines that emulate a real instruction set architecture \citep{smithvmarticle}. Instead, they execute bytecode for a machine imagined by the developer which is typically more sophisticated and generic than any real machine. This imaginary machine more easily supports the high-level nature of typical interpreted languages and---crucially---is not tied to any specific real architecture. The practice of translating a high level interpreted language into an intermediary bytecode also helps modularize the structure of the interpreter by separating the language and context analysis structures from the host-specific implementation details \citep{structureinterpreters}.
	
		Traditionally, the high level virtual machines implemented in interpreters are stack machines. Stack machines are used over register machines for a few reasons. Instructions for stack machines are smaller, because instruction operands are implicit. This means that instruction fetching is faster. It is easier to compile code for a stack machine than a register machine because the compiler does not need to implement a register allocator \citep{caseregistervm}.
	
		On the other hand, register-based machines have some advantages over stack-based machines. For instance, many tasks can be performed in fewer register instructions than stack instructions \citep{caseregistervm}. Most real machines are register machines, so there is more opportunity for register virtual machines to take advantage of the host machine hardware machine. 
		
		Ultimately, however, executing bytecode on any virtual machine is significantly slower than executing native code \citep{optimizingindirectbranch}. Just-in-time (JIT) compilation tries to bring ``the best of both worlds'' together. Modern JIT interpreters begin with the normal interpretation process, but while interpreting, profiles the executing bytecode to determine which parts of the bytecode would most benefit from native execution. Once it has identified these sections of bytecode, it compiles them into the host machine's instruction set and executes them natively instead of interpreting them \citep{historyjit}. 
		
	\section{Problem Description}
		Although JIT compilation is a good strategy to improve the performance of virtual machines, they have disadvantages in certain use cases. For instance, in multi-instance programs, where many identical threads of the same code are executed simultaneously. A real-world example of such a program is a web server, which spawns several identical program threads to serve web clients asynchronously.
		
		Native multi-instance programs share a read-only code space between instances. This is not possible in JIT interpreters because a JIT compiler alters the program code as it runs. A JIT compiler cannot do this with a shared code space because an in-progress JIT compilation might interfere with a thread executing the same part of the program. The alternative is to maintain a separate code space for each process, which is wasteful because it means that the JIT compiler will compile interpreted code separately for every running instance---essentially compiling the same code over and over. It is useful to explore opportunities for improvement in JIT-less interpreters so that situations where JIT compilation is inappropriate are not neglected. With this in mind, ``traditional'' high level virtual machines will be considered.
		
		As mentioned in the previous section, most high level virtual machines are stack machines. However, \cite{stackregistershowdown} found that an efficient register virtual machine can execute benchmarks 32\% faster than an analogous stack machine. 
		
		Because most hosts are register machines, it would seem like a good idea to try to map virtual registers to real host registers. Unfortunately, unlike virtual registers that are implemented in host memory, real registers cannot be accessed by reference \citep{caseregistervm}. This complicates writing host implementations of virtual machine instructions because it means that a different version of implementing code must be written for every combination of virtual machine operands. Despite these difficulties, it is worthwhile to investigate designs of JIT-less register virtual machines that attempt to narrow the divide between the host and the guest architecture.
	
	\section{Purpose and Scope}
		The goal of this project is to investigate the feasibility of register virtual machines for interpreters running on a modern architecture. To this end, a high-level register virtual machine will be designed, implemented and evaluated.
	
		This virtual machine will have some unique design details that have not before been investigated or thought feasible. It will map some virtual machine registers to real registers. The virtual machine will perform dispatch not only on the opcodes of instructions, but on the entire instruction word including operands. The instruction word will be dispatched to a code table containing a subroutine for every combination of virtual machine opcode and operands. 
		
		Such a code table is likely to be very large, since it grows polynomially with the number of virtual opcodes and virtual machine registers. To remain practical, a method will be developed to generate implementation code automatically for every combination of virtual opcode and operands. 
	
		An instruction set architecture will be designed which will be tailored for such a dispatch method. Naturally the goal will be to keep the number of unique instructions down, so that this code table is as small as possible, while remaining non-trivial.
		
		The virtual machine will target the Intel 64 host machine architecture, and will be written mostly in assembly, with some operating system-facing code written in C for ease of development. 
		
		To aid in evaluation of the design, a second, more traditional register virtual machine will also be implemented that supports an identical instruction set. This will be called the control virtual machine, where the other virtual machine is the experiment. The aim of the control virtual machine is to be a baseline comparison for the new virtual machine, hence its name. For the sake of fair comparison, the control will be as similar as possible to the experiment machine, except in the main detail of interpreting instructions, where the control machine will deviate and implement a traditional indirect threading dispatch, and will store virtual registers in an array.
		
		To compare these two machines, several benchmark programs will be used to test the machines. These benchmarks will be chosen to test a realistically wide range of runtime behaviours, such as integer arithmetic, string manipulation and data structure traversal. Each benchmark will be hand-written in the instruction set of the two machines. The results of these benchmarks should provide some meaningful information about the appropriateness of the new design over traditional register virtual machines.
		
		\subsection{Limitations}
			The primary usefulness of virtual machines is that they can be ported to other architectures, and can be expected to perform similarly well on those other architectures. Unfortunately, however, there is only time to implement the virtual machine design on a single architecture, albeit a very popular one. To ensure the usefulness of this paper, a general comment about how the virtual machine is expected to perform on other architectures will be made, and a few suggestions of important considerations for implementation on architectures that are sufficiently  different from Intel 64 will be presented.
				
	\section{Overview of Treatise Structure}
		\todo{Will have to wait until the rest of the treatise exists for this!}
	
% Optional
%\chapter{Requirements Gathering}
%	\todo{Leaning towards removing this section}

% 10-12 pages
\chapter{Problem Exposition}
	\section{Literature Review}
		\cite{structureinterpreters} reflect on why spending research and engineering effort on optimising interpreters is a worthwhile endeavour. Creating a host machine independent abstraction layer between original source code and machine architecture greatly simplifies writing compilers that must generate code for many platforms. This is because for every additional platform that must be supported, a separate version of the compiler targeting that platform. In a design incorporating an interpreter, only the interpreter portion needs to be rewritten for different platforms, (or perhaps not even the interpreter if a high-level language is used to implement the interpreter) and the compiler remains the same. Interpreters tend to be simple compared to compilers, so this advantage is quite significant.
		
		Interpreters offer opportunity for more powerful usability features like debuggers compared to machine code, and have slower compilers, impacting on the write-compile-test cycle of modern software development methods.
		
		\cite{caseregistervm} mention that the first successful virtual machine interpreter ran stack-based P-code for Pascal. This was the first in a line of stack-based bytecodes for popular languages like Smalltalk, Java and the Common Intermediate Language. (CIL) This suggests that there might be historic factors involved in the prevalence of stack based virtual machines.
		
		\cite{fastjava} believes that interpreters are not optimised for modern architectures features like pipelining, branch prediction and caches. They also provide encouragement by saying, ``if interpreters can be made much faster, they will become suitable for a wide range of applications that currently need a JIT.''
		
		%mark
		
		The practice of mapping virtual machine resources to host machine registers has been investigated for stack machines by \cite{stackcaching}. Ertl measured the performance of stack machines which kept the top few elements of the stack in registers. He found that direct mapping of stack positions to machine registers was only useful for the case where just the top element of the stack was cached in a register. Further caching resulted in too many load-store instructions to shuffle register values around when positions on the stack change.
		
		Ertl also attempted a technique he calls ``dynamic stack caching'' where stack elements are cached in whatever register is convenient, and an internal state machine remembers the mapping between stack positions and registers. Many versions of code to emulate each virtual machine instructions exist; the version which is executed depends on the current state of the state machine. Ertl found that this technique halves the cost of operand fetching for each additional register involved in the caching, but adds an instruction dispatch cost because of the increased number of states.
		
		In the same paper, Ertl writes the first mention of the idea of achieving register mapping between a register virtual machine and a host register machine through a table of implementations for each combination of opcode and operands, but quickly dismisses it because it would ``cause code explosion, and will probably suffer a severe performance hit on machines with small first-level caches.'' He then went on to use the R4000 MIPS processor as an example, having 8 KB of level one (L1) instruction cache.
	
		However, this approach deserves further investigation, especially with the advent of modern processors with significantly larger cache sizes---for example, 4th generation Intel Core processors, which have 32 KB of L1 instruction cache \citep{haswellarch}, four times as much as the R4000.
	
		\todo{JIT - This is the approach taken by many popular interpreters, such as the Java Virtual Machine and Microsoft's Common Language Runtime.}
	\section{Difficulties}
		Interpreters spend most of their execution time on instruction dispatch \citep{modernarchvm}. The culprit is usually an indirect branch to the implementation code for the currently executing virtual machine instruction \citep{optimizingindirectbranch}. Every interpreter has at least one indirect branch in dispatch code \citep{modernarchvm}. Because these indirect branches are in the instruction dispatch portion of the interpreter, they are executed for every virtual machine instruction. This closely couples interpreter speed to efficiency of branching.
		
		Modern architectures tend to have long pipelines and perform branch target prediction to load the target of an upcoming branch before it is executed. Branch target misprediction is very expensive because the execution of a branch happens late in the pipeline but affects the start of the pipeline \citep{optimizingindirectbranch}. Traditional interpreter designs that were optimal on older architectures perform poorly on modern pipelined architectures because branch prediction accuracy is a big factor in performance. These interpreter designs hide the logic governing branching patterns from the branch predictor. \cite{yeti} explains this by saying ``the control transfer from one body to the next is data dependent on the sequence of instructions making up the virtual program.'' This is an atypical scenario for modern processors, and they are not designed optimally for this task.
		
		To be more concrete, a popular method to predict indirect branch targets is a branch prediction buffer. (BTB) BTB makes the assumption that the targets of indirect branches are unlikely to change \citep{yeti}, and maintains a table of the last target of each branch instruction. When the branch is encountered again, BTB predicts the target will remain the same. However, this is not the case for interpreter dispatch, because the indirect branch is based on the current virtual machine opcode, which could be any one of the instructions in the virtual machine instruction set. For each of these opcodes, the interpreter will branch to a particular section of code to emulate the virtual machine instruction. This results in virtual machine instructions ``fighting'' for the single target record in the BTB that corresponds with the indirect branch.
		
		\cite{structureinterpreters} found that the branch misprediction penalty can cause up to half of the execution time on some interpreters. This high proportion is a result of the fact that most of an interpreter's time is spent in the instruction dispatch process.
		
		One way to get around this problem is by duplicating the indirect branch, resulting in a separate BTB entry for each branch instruction. This is usually done by duplicating the dispatch code at the end of each virtual instruction implementation, instead of jumping back to a common dispatch code.\todo{this isn't my idea}
		 
		
		\todo{Okay things I want to talk about:
		\begin{itemize}
			\item register machines spend a lot of time getting arguments (not a prob for us - should it be here?)
			\item indirect branches and its effect on prediction accuracy. More indirect branches means more entries in target buffer. See yeti - ``the branching patterns of the virtual program are hidden from the branch prediction resources of the underlying real processor...'' and also ``commonly used dispatch techniques represent an unusual workload for modern processors'' ... ``techniques that were efficient on older machine may be very slow on modern processors for which accurate branch prediction is essential for good performance''
			\item instruction caching?
		\end{itemize}}
% 10-12 pages
\chapter{Solution Design}
	When designing the virtual machine architecture, it will be beneficial to remember this succinct heuristic by \cite{structureinterpreters}: ``Well-designed VMs are tailored for both easy compilation from the source language and fast interpretation on real machines.''\todo{too fluffy?}
	\todo{Biggest factor for performance is the design, so we spent a lot of time on it}

\chapter{Evaluation Methods and Results}

% Last chapter
\chapter{Conclusion}
	
	\section{Opportunities for Future Development}
	
	\section{Reflection}

% Bibliography
\bibliographysection

\end{document}